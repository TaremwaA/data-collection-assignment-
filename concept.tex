% !TEX TS-program = pdflatex
% !TEX encoding = UTF-8 Unicode

% This is a simple template for a LaTeX document using the "article" class.
% See "book", "report", "letter" for other types of document.

\documentclass[11pt]{article} % use larger type; default would be 10pt

\usepackage[utf8]{inputenc} % set input encoding (not needed with XeLaTeX)

%%% Examples of Article customizations
% These packages are optional, depending whether you want the features they provide.
% See the LaTeX Companion or other references for full information

%%% PAGE DIMENSIONS
\usepackage{geometry} % to change the page dimensions
\geometry{a4paper} % or letterpaper (US) or a5paper or....
% \geometry{margin=2in} % for example, change the margins to 2 inches all round
% \geometry{landscape} % set up the page for landscape
%   read geometry.pdf for detailed page layout information

\usepackage{graphicx} % support the \includegraphics command and options

% \usepackage[parfill]{parskip} % Activate to begin paragraphs with an empty line rather than an indent

%%% PACKAGES
\usepackage{booktabs} % for much better looking tables
\usepackage{array} % for better arrays (eg matrices) in maths
\usepackage{paralist} % very flexible & customisable lists (eg. enumerate/itemize, etc.)
\usepackage{verbatim} % adds environment for commenting out blocks of text & for better verbatim
\usepackage{subfig} % make it possible to include more than one captioned figure/table in a single float
% These packages are all incorporated in the memoir class to one degree or another...

%%% HEADERS & FOOTERS
\usepackage{fancyhdr} % This should be set AFTER setting up the page geometry
\pagestyle{fancy} % options: empty , plain , fancy
\renewcommand{\headrulewidth}{0pt} % customise the layout...
\lhead{}\chead{}\rhead{}
\lfoot{}\cfoot{\thepage}\rfoot{}

%%% SECTION TITLE APPEARANCE
\usepackage{sectsty}
\allsectionsfont{\sffamily\mdseries\upshape} % (See the fntguide.pdf for font help)
% (This matches ConTeXt defaults)

%%% ToC (table of contents) APPEARANCE
\usepackage[nottoc,notlof,notlot]{tocbibind} % Put the bibliography in the ToC
\usepackage[titles,subfigure]{tocloft} % Alter the style of the Table of Contents
\renewcommand{\cftsecfont}{\rmfamily\mdseries\upshape}
\renewcommand{\cftsecpagefont}{\rmfamily\mdseries\upshape} % No bold!

%%% END Article customizations

%%% The "real" document content comes below...

\title{INFORMATION ACCESSIBILITY TO TOURIST IN UGANDA}
\author{The Author}
%\date{} % Activate to display a given date or no date (if empty),
         % otherwise the current date is printed 

\begin{document}
\maketitle

\section{Introduction}
Access of information plays a very be role in key decision making.The tourist industry has a variaty of activities to offer thus the need for a proper platform to avail information about the industry. 


\section{Background.}
The tourism industry has registered an impressive growth with new tourism activities being brought in which tourist can engage. 
However some of the information regarding these activites might at times be outdated hense affecting the decisions made by tourists.  

\section{Problem Statement}
With the growth of the tourism industry there has been a growth or increase in the information about the industry. some of the information provided to the tourist might not be upto date due to its ever changing nature.  

 \Section{Objectives} 
 The objective if this research to find out how ICT can help tourist get upto date information about the industry and understanding who this information affects the decisions made by tourist. 
 \subsection{Other Objectives }
 To collecet data from different tourist\\ 
 To anaylse the given data
\subsection{Approach/Methodology} 
The collection of data is to be done using ODK (open data kit) which entails odk collect and odk aggregate.\\
the data to be collected is:\\ numerical\\ text\\media.\\ 
The data to be collected will include \\-Platforms used by the tourist\\ Challenges faced\\ Images of places visited.


\subsection{Impact and Outcomes}
The intended outcome of this research is to properly understand how information is access by tourist and how to best improve this process.

\end{document}
